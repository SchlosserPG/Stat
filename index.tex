https://schlosserpg.github.io/Stat/


Perform and interpret hypothesis tests. Use the t.test() function to calculate the confidence interval. Discuss how confidence intervals provide a range of values that likely include the population parameter.

Understand the difference between correlation and causation. Use functions like cor(), cor.test(), and ggplot(). Reflect on the importance of considering other factors before concluding causation.

Perform a simple linear regression analysis and interpret the results. Use the lm() and summary() functions for regression analysis. Discuss the interpretation of the intercept and slope, and the meaning of R-squared.

 Conduct an ANOVA test to compare group means. Use the aov() and anova() functions for ANOVA. Discuss the F-statistic and p-value, and what they indicate about the differences between group means.

Create informative and visually appealing data visualizations. Use the ggplot2 package and functions like ggplot(), geom_bar(), geom_point(), and theme(). Emphasize the importance of clear and informative visualizations in data analysis.

Calculate and interpret confidence intervals. Use the t.test() function to calculate the confidence interval. Discuss how confidence intervals provide a range of values that likely include the population parameter

